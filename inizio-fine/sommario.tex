% !TEX encoding = UTF-8
% !TEX TS-program = pdflatex
% !TEX root = ../tesi.tex

%**************************************************************
% Sommario
%**************************************************************
\cleardoublepage
\phantomsection
\pdfbookmark{Sommario}{Sommario}
\begingroup
\let\clearpage\relax
\let\cleardoublepage\relax
\let\cleardoublepage\relax

\chapter*{Sommario}

Il presente documento descrive il lavoro svolto durante il periodo di stage, della durata di circa trecentoventi ore, dal laureando Alberto Crivellari presso l'Azienda SINAPSI INFORMATICA S.R.L..
Gli obiettivi da raggiungere erano molteplici:
\begin{itemize}
    \item In primo luogo era richiesto lo studio del gestionale SAP Business One e webservices che interagiscono con esso;
    \item In secondo luogo era richiesto lo sviluppo di un add-on da applicare sulla maschera di un modulo del SAP:
        \begin{itemize}
            \item Quest'add-on consiste nell'aggiunta di una funzionalità di stampa di un form del SAP su file esterno oppure stampa come finestra di messaggio, in una finestra all'interno del client.
        \end{itemize}
    \item Terzo ed ultimo obiettivo era l'ampliamento di parti dei webservices, in particolare 3 diversi ampliamenti:
        \begin{itemize}
            \item Aggiunta di un campo "idext", rappresentante l'id esterno di un intervento;
            \item Aggiunta nel webservice di aggiunta intervento di un campo "extraj\_module", rappresentante informazioni aggiuntive sull'intervento, sotto forma di stringa;
            \item Aggiunta del campo "extraj\_module", anche nel webservice di lettura degli interventi.
        \end{itemize}
\end{itemize}


Gli obiettivi e risultati raggiunti dall'applicazione sono stati considerati più che sufficienti dall'azienda.
In particolare sono stati raggiunti tutti gli obiettivi obbligatori concordati nel piano di lavoro.

Purtroppo non è stato possibile effettuare test automatici del codice, poichè il codice non è abbastanza corposo da renderli necessari.
%\vfill
%
%\selectlanguage{english}
%\pdfbookmark{Abstract}{Abstract}
%\chapter*{Abstract}
%
%\selectlanguage{italian}

\endgroup			

\vfill

