% !TEX encoding = UTF-8
% !TEX TS-program = pdflatex
% !TEX root = ../tesi.tex

%**************************************************************
\chapter{Tecnologie coinvolte}
\label{cap:tecnologie-coinvolte}
%**************************************************************
\intro{In questo capitolo approfondiremo l'ampliamento dei webservices PHP, con protocollo SOAP}
%**************************************************************
\section{PHP}
\begin{flushleft}
	Insieme ad Apache viene utilizzato per sviluppare applicazioni web lato server.
	
	In questo caso, viene utilizzato dall'azienda per lo sviluppo e gestione dei webservice sui server \gls{aws}.
\end{flushleft}
\begin{figure}[!h] 
	\centering
	\includegraphics[scale = 1]{immagini/tecnologie/apache}
\end{figure}
\begin{figure}[!h] 
	\centering
	\includegraphics[scale = 0.3]{immagini/tecnologie/php}
\end{figure}
\section{VB.NET}
Linguaggio di programmazione per programmare gli add-ons, su Visual Studio, utilizzando le librerie \gls{sapb1}.
\begin{figure}[!h] 
	\centering
	\includegraphics[scale = 0.4]{immagini/tecnologie/vb-header}
\end{figure}
\section{C\#}
\begin{flushleft}
	Linguaggio di programmazione alternativo per programmare gli add-ons.
	
	Sempre su Visual Studio, utilizzando le librerie fornite dalla \gls{sapb1} SDK.
\end{flushleft}
\begin{figure}[!h] 
	\centering
	\includegraphics[scale = 0.4]{immagini/tecnologie/csharp}
\end{figure}
\section{SoapUI}
\begin{flushleft}
	Applicazione utilizzata per testare ed interagire con i webservices PHP, con protocollo \gls{soap}.
	
	E' possibile inserire una chiamata e l'applicazione ritorna una risposta dopo aver interagito con i webservice.
	
\end{flushleft}
\begin{figure}[!h] 
	\centering
	\includegraphics[scale = 1]{immagini/tecnologie/soapui}
\end{figure}
\section{HeidiSQL}
\begin{flushleft}
	Strumento di gestione di databases, quali MySQL, MariaDB, InnoDB.
	
	In questo caso utilizzato per il database MySQL del server \gls{aws}, per controllare che le operazioni di webservices modifichino o meno il database.
\end{flushleft}
\begin{figure}[!h] 
	\centering
	\includegraphics[scale = 0.5]{immagini/tecnologie/HeidiSQL}
\end{figure}
\section{Microsoft Visual Studio}
\begin{flushleft}
	Strumento di sviluppo software, \gls{ide}, utilizzato per programmare codice di vario tipo, da applicazione desktop ad app mobile, e raramente utilizzato per applicazioni web.
	
	In questo caso necessario per programmare gli add-ons, poichè le librerie necessarie sono predisposte per Visual Studio.
\end{flushleft}
\begin{figure}[!h] 
	\centering
	\includegraphics[scale = 0.4]{immagini/tecnologie/visual-studio}
\end{figure}
\section{Microsoft SQL Server Management Studio (SSMS)}
\begin{flushleft}
	Applicazione utilizzata per gestire i database, viene utilizzata particolarmente per i server SQL Server.
	
	Nel nostro caso l'applicazione viene utilizzata per accedere direttamente ai database del server SAP.
	
\end{flushleft}
\begin{figure}[!h] 
	\centering
	\includegraphics[scale = 0.5]{immagini/tecnologie/microsoft ssms}
\end{figure}

\newpage

\section{Remote Desktop}
\begin{flushleft}
	Applicazione preinstallata su Windows per accedere ad altri computer nella stessa rete locale.
	
	Viene utilizzata per accedere ai server SAP, presenti nella rete locale aziendale.
\end{flushleft}
\begin{figure}[!h] 
	\centering
	\includegraphics[scale = 0.5]{immagini/tecnologie/remote-desktop}
\end{figure}

\section{Domino Lotus Notes}
\begin{flushleft}
	Applicazione utilizzata principalmente per la mail aziendale.
	
	Contiene però altre funzionalità quali un calendario aziendale e un server condiviso (domino) aziendale, che comprende database di fatture e password per accedere a diversi software aziendali o di clienti dell'azienda.
\end{flushleft}

\begin{figure}[!h] 
	\centering
	\includegraphics[scale = 0.5]{immagini/tecnologie/domino-lotus-notes.png}
\end{figure}

\section{Postman}
\begin{flushleft}
    Applicazione utilizzata per gestire e testare le API.
    
    Nel caso dell'azienda è stata utilizzata principalmente per controllare il corretto funzionamento e testare nuovi tipi di richieste sui webservices REST API di \gls{sapb1}.
\end{flushleft}

\begin{figure}[!h] 
	\centering
	\includegraphics[scale = 0.5]{immagini/tecnologie/postman}
\end{figure}