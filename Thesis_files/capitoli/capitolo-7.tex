% !TEX encoding = UTF-8
% !TEX TS-program = pdflatex
% !TEX root = ../tesi.tex

%**************************************************************
\chapter{Conclusioni}
\label{cap:conclusioni}
%**************************************************************
\intro{In questo capitolo finale viene verificato il raggiungimento degli obiettivi prestabiliti e vengono fatte delle considerazioni sullo stato attuale del programma}
%**************************************************************
\section{Raggiungimento degli obiettivi}


Gli obiettivi e risultati raggiunti dall'applicazione sono stati considerati più che sufficienti dall'azienda.

In particolare sono stati raggiunti tutti gli obiettivi obbligatori concordati nel piano di lavoro.

Purtroppo non è stato possibile effettuare test automatici del codice, poichè il codice non è abbastanza corposo da renderli necessari.

I file prodotti e la stampa su finestra di messaggio sono risultati conforme alle aspettative e ne è stata verificata la correttezza.



%**************************************************************
\section{Conoscenze acquisite}
Le conoscenze acquisite sono state molteplici, a partire da nuovi linguaggi di programmazione al mondo dei gestionali e a tutte le applicazioni di supporto che permettono di far funzionare tutto il sistema con efficienza.

In particolare, le conoscenze acquisite principalmente sono:
\begin{itemize}
    \item \textbf{Conoscenze sul gestionale SAP Business One: }Uno studio approfondito della documentazione SAP presente online, e uno studio parallelo su ambienti di test SAP, per testare le conoscenze via via acquisite dalla documentazione;
    \item \textbf{Conoscenze acquisite su applicazioni di supporto: } Molte nuove conoscenze su applicazioni come SQL Server Management Studio, Postman, SoapUI, HeidiSQL e Remote Desktop per gestire le varie parti dell'infrastruttura presistente;
    \item \textbf{Conoscenze acquisite su C\# e librerie di SAP relative: } Uno studio della \textit{Software Development Kit}, in breve SDK, fornita da SAP Business One, ovvero le librerie SAP per programmare gli add-ons, e l'applicazione di queste librerie su codice in C\#;
    \item \textbf{Conoscenze acquisite su PHP: } L'aver ampliato notevolmente le conoscenze nel linguaggio PHP, dato il lavoro di ampliamento su un codice già complesso e strutturato, con necessario studio e comprensione profonda di questo codice preesistente.
\end{itemize}

%**************************************************************
\section{Esperienza di stage}

Nel complesso l'esperienza di studio dell'infrastruttura preesistente e sviluppo dell'add-on si è rivelata molto formativa.

Ma senza ombra di dubbio l'ampliamento delle funzioni PHP per i webservices è stata altamente formativa.

\newspace

Sono soddisfatto di questo progetto, poichè sono riuscito a comprendere un po' il mondo dei gestionali e soprattutto quanta conoscenza ci sia ancora da apprendere su questo campo.

Sono soddisfatto della parte tecnologica dell'attività di stage in quanto mi ha permesso di imparare C\# e approfondire il linguaggio PHP, e di applicare le mie conoscenze di SQL.

\newspace

Ho potuto godere di un'ampia autonomia nella gestione ed organizzazione del mio lavoro, nel rispetto di vincoli e funzionalità stabilite ad inizio progetto.

Grazie a questa esperienza di stage ho potuto integrarmi e confrontarmi con una realtà molto diversa da quella che ero abituato a vedere e relazionarmi con persone che vedono le cose sotto un altro punto di vista.

Particolarmente importante è stata la collaborazione con i capi progetto e con i vari dipendenti dell'azienda. \'E stato essenziale il supporto che mi hanno fornito nell'ambientamento e nella comprensione dei nuovi argomenti, differenti da quelli studiati nel mio percorso scolastico.
